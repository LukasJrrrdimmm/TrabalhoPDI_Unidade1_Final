\documentclass{article}
\usepackage[utf8]{inputenc}
%\usepackage[brazil]{babel}

\title{Trabalho de Processamento Digital de Sinais}
\vspace{5cm}
\author{Departamento de Computação e Automação - DCA/UFRN\\
Engenharia da Computação UFRN 2018.1\\
Aluno/Discente: Lukas Maximo Grilo Abreu Jardim\\
Professor/Docente: Agostinho de Medeiros Brito Júnior}

\begin{document}
	\maketitle
	\newpage
	\section{Apresentação Inicial}
	\vspace{5mm}
	Nesse trabalho será apresentado alguns exemplos de manipulação de imagens bem como os esforços que visam melhorar sua definição e provocar alguns efeitos extras em sua exibição. E portanto, este trabalho terá foco em exibir aplicações de demais técnicas de manipulaçao de imagens vistas em sala de aula.
	\newpage
	\section{exibição da imagen (Open CV)}
	O arquivo pixels.cpp foi alterado para manipular boa parte dos pixels da imagem ao invés de um,permitindo a inverter o preto e o branco, colorir a imagem, e posteriormente, a necessidade de instalar um filtro para processa-la e isolar algumas regiões específicas.
	\vspace{5mm}
	\section{manipulação de imagem GIMP}
	\vspace{5mm}
	Utilizando o GIMP foi manipulado o arquivo Imagem.jpeg para tentar uma melhoria em sua definição através de 2 filtros normalizados de detecção de borda e um filtro de aguçamento (3 camadas de filtragem).
	imagem: lobo(modificado).png
	\vspace{5mm}
	\section{trab.cpp}
	\vspace{5mm}
	Nessa parte do trabalho o objetivo é calcular o histograma de uma imagem (graywolf.jpg), exibindo a imagem, + o histograma respectivo de cada cor.\\
\end{document}